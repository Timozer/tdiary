%!TEX TS-program = xelatex
% vim: set fenc=utf-8

% -*- coding: UTF-8; -*-
%!TEX encoding = UTF-8 Unicode 

周六

学习笔记:

图片被识别成什么不仅仅取决于图片本身,还取决于图片是如何被观察的。

在读于建国的卷积神经网络的时候,图片识别那块儿,突然有个奇怪的想法,人类
所看到的世界是否是真实的世界,还是说只是适合人类生存的视觉联觉。

如何在图片识别中加入远近距离(多次采样,转换视角,补全信息)的探测,
和光照强弱方向等。

CNN 通过局部连接达到空间共享,聪儿实现画面不变性(invariance)。

ResNet:跨层连接,batch normalization

GoogLeNet:Inception-v4 模块的使用(尺寸不变性)

DenseNet:将跨层从头进行到尾

趋势:
\begin{itemize}
    \item 使用 small filter 和 pooling
    \item 去掉 parameters 的全连接
    \item Inception
    \item 跳层连接
\end{itemize}
